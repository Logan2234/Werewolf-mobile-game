\documentclass[10pt]{article}
%%% Pour le français %%%
\usepackage[utf8]{inputenc}
\usepackage[T1]{fontenc}
\usepackage[french]{babel}
%%%%%%%%%%%%%%%%%%%%%%%%
\usepackage{fancyhdr} % En-tête et pied de page personnalisés
% \usepackage{listings} % Pour du beau code coloré
% \usepackage{minted} % Pour du Ocaml coloré
%%%% Pour des maths %%%%
\usepackage{mathtools} % Pour pleins de commandes avec des maths
\usepackage{amssymb} % Pour les symboles
\usepackage{amsmath} % Pour les environnements splits, equations, align etc..
\usepackage{amstext} % Pour utiliser \text
% \usepackage{mathrsfs} % 3 fonts pour les 26 lettres
% \usepackage{amsthm} % Custom des theoremes
% \usepackage{tikz} % Pour des graphiques
% \usepackage{stmaryrd} % Pour les double crochets, parenthèses etc..
%%%%%%%%%%%%%%%%%%%%%%%%
% \usepackage{layout} % Pour afficher le gabarit de mise en page
% \usepackage{geometry} % Pour régler les marges
% \usepackage{setspace} % Pour modifier l'interligne
% \usepackage{ulem} % Pour souligner et barrer du texte
%%% Pour des polices %%%
% \usepackage{bookman}
% \usepackage{charter}
% \usepackage{newcent}
% \usepackage{lmodern}
% \usepackage{mathpazo}
% \usepackage{mathptmx}
%%%%%%%%%%%%%%%%%%%%%%%%
% \usepackage{url} % Pour citer des urls
% \usepackage{graphicx} % Pour travailler sur des images
% \usepackage{color} % Pour manipuler les couleurs et colorer le texte
%%% Renewed commands %%%
\def\R{\mathbb{R}}
\def\C{\mathcal{C}}
\def\P{\mathbb{P}}
\def\E{\mathbb{E}}
\def\R{\mathbb{R}}
\def\ind{\mathbb{1}}
%%%%%%%%%%%%%%%%%%%%%%%%
\title{Manuel utilisateur}
\author{BERGONZOLI Maud - LURI VANO Jorge - WILLEM Logan}
\date{}

\pagestyle{fancy}

\chead{Manuel utilisateur}
\lhead{}
\rhead{}
\lfoot{}
\cfoot{}
\rfoot{\thepage}

\begin{document}

\maketitle
\tableofcontents

\newpage

\section{Introduction}

L'application est une variante mobile et en ligne du jeu de Loup-Garou.

L’action se déroule dans un village dont certains habitants sont humains et d’autres des loups-garous. Chaque nuit, les loups-garous se transforment et tuent un villageois. Pour se défendre, chaque jour, les villageois éliminent l’un d’entre eux. Le but de chaque faction est d’éliminer l’autre, mais les humains ne savent pas qui sont les loups-garous avant la fin de la partie.

Au début d’une partie, chaque joueur se voit attribuer un rôle: loup-garou ou humain. Ces rôles sont cachés aux autres joueurs. Les loups-garous et les humains sont collectivement appelés « villageois » (en effet les loups-garous sont aussi des habitants du village).

Lorsque le villageois d’un joueur est tué, ce joueur est éliminé. Les joueurs éliminés ne peuvent plus intervenir mais ont accès en lecture à toutes les discussions du jeu (sur la place du village comme dans le repaire des loups-garous), y compris les discussions archivées.

Le jeu fonctionne différemment le jour et la nuit. Par défaut le jour dure de 8h00 à 22h00 et la nuit de 22h00 à 8h00. La durée des journées est configurable lors de la création d'une session.



\section{Utilisateurs visés}

Cette application est un jeu mobile. Elle fonctionne sur Android et iOS.
De ce fait, l'application est destinée à un public de joueurs mobile.
Elle convient à des joueurs de loup-garou passionnés, voulant trouver une version du jeu en ligne gratuite et accessible. Elle convient plus généralement aux amateurs de jeux en ligne mais aussi à tout adolescents et adultes aptes à jouer à des jeux de stratégie et souhaitant passer du bon temps avec ses proches.

\section{Interface utilisateur}

\subsection{Les différentes vues}
\subsection{Page d'inscription}
\subsection{Page de connexion}
\subsection{Page d'accueil}
\subsection{Page de création de parties}
\subsection{Page pour rejoindre une partie}
\subsection{Page d'attente}
\subsection{Page de jeu}
\subsubsection{Page d'informations}
\subsubsection{Page de discussion}
\subsubsection{Page de vote}
\subsubsection{Page des joueurs}

\section{Fonctionnalités}
\subsection{Inscription et connexion}
\subsection{Création d'une partie}
\subsection{Joindre une partie}
\subsection{Fonctionnalités disponibles en attente}
\subsubsection{Consulter les informations de la partie}
\subsubsection{Consulter la liste des joueurs}
\subsection{Fonctionnalités en jeu}
\subsubsection{Consulter les informations}
\subsubsection{Utiliser son pouvoir}
\subsubsection{Choisir un salon de discussion et discuter}
\subsubsection{Voter}
\subsubsection{Consulter la liste des joueurs}

\section{Faiblesses de l'application}

\subsection{Choix des dates}

Le principe de choix d'une date et d'une heure de début à l'aide d'un selecteur de date et d'heure n'est fonctionnel que sur Android. Malheureusement, le principe utilisé n'est pas compatible avec les appareils iOS.

\subsection{Retour en arrière}

La possibilité de retourner en arrière en fonction des vues n'est disponible que sur certains appareils Android, où le bouton de retour est affiché en bas de l'écran. De ce fait, les utilisateurs iOS ne disposant pas de ce bouton ne peuvent retourner en arrière de vues en vues.

\section{FAQ}


\end{document}